\chapter{Conclusions and Discussion}
\label{ch:conclusion}

% the code below specifies where the figures are stored
\ifpdf
    \graphicspath{{conclusion/figures/PNG/}{conclusion/figures/PDF/}{conclusion/figures/}}
\else
    \graphicspath{{conclusion/figures/EPS/}{conclusion/figures/}}
\fi

% ----------------------------------------------------------------------
%: ----------------------- conclusion content ----------------------- 
% ----------------------------------------------------------------------

\section{Contributions}
In this thesis, we modeled functional networks of epileptic neocortex in human patients and studied functional pathways that drive the generation, propagation, and termination of seizures. The primary goal of this thesis was to characterize epilepsy as a network disorder of dysfunctional brain circuitry by abstracting beyond current clinical practice of localizing isolated islands of pathologic cortical tissue. Addressing this goal might relieve the clinical burden of identifying targets for surgical intervention, which in many cases leads to little reduction in a patient's seizures, and may pave the road for integrating novel neurotechnologies capable of controlling network dysfunction into clinical practice. To this end, we developed and applied graph theoretic and machine learning algorithms for objectively mapping functional network pathways between distributed cortical structures related to conventional clinically-defined targets. 

In Chapter \ref{ch:netdrivers}, we modeled time-varying functional connectivity of the epileptic network and applied a novel community detection algorithm for clustering patterns of connection geometry into discrete, time-dependent network states preceding and during seizures. We found that the network states corresponded to stages of seizure generation, propagation and termination and connections between cortical structures in the seizure-onset zone are persistently stronger than all other network connections. Results from these investigations suggest that clustering time-varying network architecture can parse seizures and robustly pinpoint seizure-onset regions, potentially improving the inter-rater reliability currently hindering interpretation of electrophysiology from patients with neocortical epilepsy.

In Chapter \ref{ch:mapsubnet}, we disentangled modular sub-networks from time-varying functional connectivity and applied an ensemble clustering technique for quantifying similarity of cortical pathways engaged during seizures and normal, interictal periods. We found that functional connections of the epileptic network form cohesive, stable sub-networks that are expressed during seizures and recapitulated during interictal periods. These sub-networks form clusters that reliably predict functional pathways incorporating clinically-defined seizure-onset regions. Secondly, we find that functional sub-networks are persistently expressed during interictal periods and more transiently expressed during seizures. Our findings have clinical implications in delineating sub-networks generating seizures during interictal periods, many hours prior to seizure onset.

In Chapter \ref{ch:selfreg}, we developed a simulation technique for conducting virtual resections on regions of the epileptic network and applied the approach to a data set of two types of seizures, those that spread over cortex and others that remain contained to a local cortical region. We found that specific network controllers in cortical regions outside clinically-defined seizure onset nodes, regulate the future spread of seizures. These findings suggest network regions outside of conventional clinical targets regulate seizure spread, suggesting the epileptic network is more complex and distributed than originally believed.

Taken together, this thesis work demonstrated that functional connectivity derived from activity of neural populations at millimeter scale explains, at least in part, functional architecture of the epileptic network. Our findings suggested that regions generating seizures exhibit stereotyped patterns of connectivity, which may be predicted many hours before epileptic events. Finally, we generated a notion of network stability that may be used for probing cortical controllers of seizure evolution.

\section{Future Studies}
While this work yields many insights regarding the architecture of the epileptic network, clinical translation of these tools requires further validation. Many of the network properties studied in this thesis were related back to ``gold-standard'' markings of seizure-onset regions in a data set of highly varying patient outcome. This poses several problems: (i) a consensus definition of seizure-onset is lacking and unreliable across clinicians and institutions, (ii) well-defined seizure-onset may still lead to poor outcome, and (iii) resection margins may only partially overlap with the seizure-onset region.

To reliably validate objective algorithms for mapping dysfunction in epileptic networks, we should differentiate functional connectivity within the resection zone of patients who experienced good and bad seizure reduction. This study would enable a more direct comparison of network architecture with a variable closest to seizure reduction. We hypothesize that spatially distributed, strongly connected nodes, which significantly corresponded to cortical structures within seizure-onset zone, may have a large overlap with resected nodes in patients with good outcome. Alternatively, we can test whether resecting controllers in regions outside the seizure-onset nodes leads to better containment of seizure activity -- verifying the utility of the developed virtual cortical resection technique. 

Along a similar line of thought, we speculate that network architecture may predict post-surgical seizure freedom. In patients with seizures generated near network hubs connecting many distributed cortical structures, we hypothesize resective surgery may not lead to favorable outcome. To test this hypothesis, we could use linear mixed effects models to stratify clinical scales of seizure freedom (Engel or ILAE) based upon network measures such as stability (synchronizability), number of sub-networks, or spatial extent of strongest connections. These studies can have potential clinical impact in screening patients who are good candidates for resective surgery.

Another line of work that requires validation is using functional architecture of the epileptic network to pinpoint stimulation targets for implantable devices. We hypothesize that functional connections within the network can be leveraged to deliver stimulation to specific cortical structures. Testing this hypothesis might involve tracking the effects of stimulation on functional architecture to predict whether stimulating at one node results in changes in neural activity at a connected node. We can then assess whether stimulating nodes with connections leading to seizure-onset areas might abort seizure evolution.

All facets of this important future work will be essential for introducing the findings of this dissertation into clinical practice. 
