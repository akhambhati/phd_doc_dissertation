% Thesis Abstract -----------------------------------------------------

\begin{abstracts}
Epilepsy is a debilitating brain disorder that causes recurring seizures in approximately 60 million people worldwide. For the one-third of epilepsy patients whose seizures are refractory to medication, effective therapy relies on reliably localizing where seizures originate and spread. This clinical practice amounts to delineating the epileptic network through neural sensors recording the electrocorticogram. Mapping functional architecture in the epileptic network is promising for objectively localizing cortical targets for therapy in cases of neocortical refractory epilepsy, where post-surgical seizure freedom is unfavorable when cortical structures responsible for generating seizures are difficult to delineate. In this work, we develop and apply network models for analyzing and interrogating the role of fine-grain functional architecture during epileptic events in human neocortical networks. We first develop and validate a model for objectively identifying regions of the epileptic network that drive seizure dynamics. We then develop and validate a model for disentangling network pathways traversed during ``normal'' function from pathways that drive seizures. Lastly, we devise and apply a novel platform for predicting network response to targeted lesioning of neocortical structures, revealing key control areas that influence the spread of seizures to broader network regions. The outcomes of this work demonstrate network models can objectively identify and predict targets for treating neocortical epilepsy, blueprint potential control strategies to limit seizure spread, and are poised for further validation prior to near-term clinical translation.
\end{abstracts}

% ---------------------------------------------------------------------- 
