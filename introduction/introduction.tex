
%: ----------------------- introduction file header -----------------------
\chapter{Introduction}

% the code below specifies where the figures are stored
\ifpdf
    \graphicspath{{introduction/figures/PNG/}{introduction/figures/PDF/}{introduction/figures/}}
\else
    \graphicspath{{introduction/figures/EPS/}{introduction/figures/}}
\fi

% ----------------------------------------------------------------------
%: ----------------------- introduction content ----------------------- 
% ----------------------------------------------------------------------

Epilepsy is a neurological disease of recurring seizures that affects an estimated 60 million people worldwide \cite{kwan2000early}. Patients experiencing a seizure may exhibit one or more of a number of clinical signs (e.g. confusion, muscle convulsion, loss of consciousness) that can be dangerous, because seizures that can manifest anywhere and anytime. In more than one-third of epilepsy patients seizures are pharmaco-resistant and increase risk of premature death, anxiety, depression, and cognitive dysfunction \cite{kwan2000early, kwan2011drug-resistant}. For drug-resistant patients, neurologists seek an understanding of what behavioral conditions precipitate the seizure and where in the brain does the seizure originate.

The inherent challenges in treating drug-resistant epilepsy is that the disease characteristics such as seizure precipitants, symptoms, and onset location are unique to each patient. Patients undergo extensive clinical work-up for mapping their epileptic brain regions, which includes imaging, for localizing anatomic malformations of cortical tissue, neuropsychological testing, for assessing dysfunction associated with epileptic activity in cognitive domains, and electrophysiology, for diagnosing dysfunctional brain activity in the seizure-onset zone. These tests enable a patient's clinical team to prescribe a variety of treatment options, of which surgically removing dysfunctional tissue is the most common \cite{kwan2011drug-resistant}. When seizures are localized to mesial temporal lobe structures such as hippocampus and amygdala, surgical resection of the anterior temporal lobe yields seizure freedom in 70\% of cases \cite{kwan2011drug-resistant}. More sobering is when seizures originate in the neocortex, long-term seizure freedom rates vary between 27--66\% of cases depending on etiology and localization \cite{tellez-zenteno2005long-term}. In this work, we focus our study on patients with neocortical epilepsy. The poor post-surgical outcome for neocortical epilepsy patients has resulted in a paradigm shift from localizing and resecting epileptic tissue towards mapping the epileptic network and identifying key regions for intervention \cite{spencer2002neural, kramer2012epilepsy, lehnertz2014evolving}.

Innovating beyond crude resective surgery, many novel neurotechnologies are being developed to target and affect specific cortical structures of brain networks \cite{stacey2008technology}. Such neurotechnology includes implantable devices that stimulate or modulate brain activity to control seizures \cite{fisher2010electrical, morrell2011responsive} and laser interstitial thermal therapy that ablates pathologic brain tissue \cite{tovar-spinoza2013use, medvid2015current}. Compared to traditional resective surgery, these treatment options afford a greater degree of precision to target dysfunction while limiting brain regions vital to normal function. While clinicians are excited about introducing novel neurotechnology into clinical practice, preliminary data shows only moderate seizure reduction on the order 40\% \cite{morrell2011responsive}. A primary burden for better performance and wide-scale clinical adoption is an inadequate understanding of how the epileptic network is organized and which cortical structures are optimal targets for therapy.

Modern clinical practice is already working towards diagnosing network abnormalities in drug-resistant epilepsy \cite{spencer2002neural}. In their clinical work-up epileptologists study the electrocorticogram (ECoG), brain activity generated by populations of local neurons, to capture signatures of epileptic activity and describe their spatial and temporal patterns of evolution. Understanding these complex dynamics help clinicians pinpoint the source of dysfunction, or seizure-onset zone, and map its spread to secondarily recruited brain regions. More formally, the cortical pathways corresponding to the generation and evolution of this pathologic activity comprise the functional architecture of the epileptic network. While diagnosing drug-resistant epilepsy under network formalisms may improve patient outcome, especially when applied in conjunction with novel neurotechnology, a non-unified definition of dysfunction in the epileptic network leads to poor inter- and intra-rater reliability in diagnosis. Thus, a central challenge in translating the notion of the epileptic network to guide clinical intervention is the objective characterization of dysfunctional pathways that contribute to the development of epileptic activity.

This dissertation is designed to address this challenge by: (i) objectively formulating a notion of the epileptic network using top-down modeling to identify functional pathways between brain regions, (ii) connecting the network model to clinical markers used for diagnosis, and (iii) predicting targets for clinical intervention based on network dysfunction. In Chapter \ref{ch:background}, we provide a background on characteristics of neocortical epilepsy and introduce a framework for modeling functional networks from electrophysiology. In Chapter \ref{ch:netdrivers}, we develop a dynamic model of the epileptic network for describing the role of distributed cortical structures in seizure generation, termination and propagation. In Chapter \ref{ch:mapsubnet}, we develop an unsupervised learning technique for disentangling cohesive sub-structures from a dynamic model of the epileptic network and use it to compare network topology between baseline and seizure states. In Chapter \ref{ch:selfreg}, we develop virtual cortical resection technique for predicting network response to the removal of specific regions and use it to compare mechanisms of seizure spread. We conclude the thesis in Chapter \ref{ch:conclusion} with a summary of our findings, a discussion of our work, contributions to epilepsy and network science, and a roadmap for future work.
% ----------------------------------------------------------------------



